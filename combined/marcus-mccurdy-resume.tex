% LaTeX source of my resume
% =========================

% Commented for easy reuse... ;)

% See the `README.md` file for more info.

% This file is licensed under the CC-NC-ND Creative Commons license.


% start a document with the here given default font size and paper size
\documentclass[10pt,a4paper]{article}

% include the `tex` instructions that takes care of loading packages and defining commands

% Copyright (c) 2012 Cies Breijs
%
% The MIT License
%
% Permission is hereby granted, free of charge, to any person obtaining a copy
% of this software and associated documentation files (the "Software"), to deal
% in the Software without restriction, including without limitation the rights
% to use, copy, modify, merge, publish, distribute, sublicense, and/or sell
% copies of the Software, and to permit persons to whom the Software is
% furnished to do so, subject to the following conditions:
%
% The above copyright notice and this permission notice shall be included in
% all copies or substantial portions of the Software.
%
% THE SOFTWARE IS PROVIDED "AS IS", WITHOUT WARRANTY OF ANY KIND, EXPRESS OR
% IMPLIED, INCLUDING BUT NOT LIMITED TO THE WARRANTIES OF MERCHANTABILITY,
% FITNESS FOR A PARTICULAR PURPOSE AND NONINFRINGEMENT. IN NO EVENT SHALL THE
% AUTHORS OR COPYRIGHT HOLDERS BE LIABLE FOR ANY CLAIM, DAMAGES OR OTHER
% LIABILITY, WHETHER IN AN ACTION OF CONTRACT, TORT OR OTHERWISE, ARISING FROM,
% OUT OF OR IN CONNECTION WITH THE SOFTWARE OR THE USE OR OTHER DEALINGS IN THE
% SOFTWARE.


% Some commands for making a LaTeX resume
% =======================================

% Commented ;)

% See the README.md file for more info



% \documentclass[10pt,a4paper]{article}  % i do this in the document itself


%%% LOAD AND SETUP PACKAGES

\usepackage[a4paper,margin=0.75in]{geometry}
\usepackage{mdwlist}   % to finetue lists with a inline heading and indented content (see Experiences)
\usepackage{multicol}  % for multiple column text
\usepackage{relsize}   % for \textscale, which I prefer over \sc (small caps), see my \acr command
\usepackage[english]{babel}
\hyphenation{Some-long-word}

\usepackage[pdftex]{hyperref}  % yups, URLs everwhere...
\usepackage{xcolor}  % ... and color them links
\definecolor{dark-blue}{rgb}{0.15,0.15,0.4}
\hypersetup{colorlinks,linkcolor={dark-blue},citecolor={dark-blue},urlcolor={dark-blue}}

\usepackage{ifxetex}
\ifxetex
  \usepackage{fontspec}
  \setmainfont
    [ ExternalLocation ,
      Mapping          = tex-text ,
      Numbers          = OldStyle ,
      Ligatures        = {Common,Contextual} ,
      BoldFont         = texgyrepagella-bold.otf ,
      ItalicFont       = texgyrepagella-italic.otf ,
      BoldItalicFont   = texgyrepagella-bolditalic.otf ]
    {texgyrepagella-regular.otf}
  % Comment out the previous statement and uncomment the following line to use the
  % Linux Libertine font (it has nice lignatures).
  % Make sure to have the `ttf-linux-libertine` package installed on Ubuntu.
%  \setmainfont[Mapping=tex-text, Numbers=OldStyle, Ligatures={Common,Contextual}]{Linux Libertine O}
\else
  \usepackage{tgpagella}  % this case we lack lower case numbers, ligatures and some typographic niceties
\fi



%%% DOCUMENT WIDE STYLING

\pagestyle{empty}
\setlength{\tabcolsep}{0em}
\xspaceskip7pt  % some more spacing between sentences (use "i.e.\ " or "with SQL\@. " in case of errors)


%%% CUSTOM COMMANDS

% main title (name) with subtitle (date)
\newcommand*\maintitle[2]{\noindent{\LARGE \textbf{#1}}\ \ \ \emph{#2}}

% title for the root sections (experience, education, etc) of the resume
\newcommand*\roottitle[1]{\subsection*{#1}\vspace{-0.3em}\nopagebreak[4]}

% acr command, to quickly mark acronyms for special formatting
\newcommand*\acr[1]{\textscale{.85}{#1}}

% pretty bullet (created from a much smaller centerdot), \sbull contains its spacing
\newcommand*\bull{\raisebox{-0.365em}[-1em][-1em]{\textscale{4}{$\cdot$}}}
\newcommand*\sbull{\ \ \bull \ \ }

% it seems not to work when simply using \parindent...
\newlength{\newparindent}
\addtolength{\newparindent}{\parindent}

% a double \parindent...
\newlength{\doubleparindent}
\addtolength{\doubleparindent}{\parindent}
\addtolength{\doubleparindent}{\parindent}

% indentsection style, used for sections that aren't already in lists
% that need indentation to the level of all text in the document
\newenvironment{indentsection}%
{\begin{list}{}%
  {\setlength{\leftmargin}{\newparindent}\setlength{\parsep}{0pt}\setlength{\parskip}{0pt}\setlength{\itemsep}{0pt}\setlength{\topsep}{0pt}}%
}
{\end{list}}

% headerrow command, used for a new employer
\newcommand{\headedsection}[3]{\nopagebreak[4]\begin{indentsection}\item[]\textscale{1.1}{#1}\hfill#2#3\end{indentsection}\nopagebreak[4]}

% subheaderrow command, used for a new position
\newcommand{\headedsubsection}[3]{\nopagebreak[4]\begin{indentsection}\item[]\textbf{#1}\hfill\emph{#2}#3\end{indentsection}\nopagebreak[4]}

% body text (indented)
\newcommand{\bodytext}[1]{\nopagebreak[4]\begin{indentsection}\item[]#1\end{indentsection}\pagebreak[2]}

% \vspace variaties
\newcommand{\breakvspace}[1]{\pagebreak[2]\vspace{#1}\pagebreak[2]}
\newcommand{\nobreakvspace}[1]{\nopagebreak[4]\vspace{#1}\nopagebreak[4]}

% \spacedhrule a horizontal line with some vertical space before and after it
\newcommand{\spacedhrule}[2]{\breakvspace{#1}\hrule\nobreakvspace{#2}}

% \inlineheadsection command, used for a new employer
\newcommand{\inlineheadsection}[2]{\begin{basedescript}{\setlength{\leftmargin}{\doubleparindent}}\item[\hspace{\newparindent}\textbf{#1}]#2\end{basedescript}\vspace{-1.7em}}

% apo command, for an apostrophe that looks good on old style nums
\newcommand{\apo}{\raisebox{-.18ex}{'}{\hspace{-.1em}}}

% non space that allows line breaks
\newcommand*{\nsp}{\hskip0pt}

%%% MORE SPECIFIC COMMANDS

% CPP command (found it in some corner of the internet and decided to use it)
\newcommand{\CPP}{C\nolinebreak[4]\hspace{-.04em}\raisebox{.20ex}{\footnotesize\bf++} }

% KTurtle command :)
\newcommand{\KTurtle}{\acr{KT}urtle }



% % these are in the document itself:
%
% \begin{document}
% ...the document text...
% \end{document}



\begin{document}  % begin the content of the document
\sloppy  % this to relax whitespacing in favour of straight margins

\maintitle{Marcus McCurdy}{}% title on top of the document

\nobreakvspace{0.3em}  % add some page break averse vertical spacing

% \noindent prevents paragraph's first lines from indenting
% \mbox is used to obfuscate the email address
% \sbull is a spaced bullet
% \href well..
% \\ breaks the line into a new paragraph
\noindent\href{mailto:marcus.mccurdy@gmail.com}
{marcus.mccurdy\mbox{}@\mbox{}gmail.com}\sbull
\textsmaller{+1}714-791-8345\sbull
\href{http://www.linkedin.com/in/marcusmccurdy}
    {www.linkedin.com/in/marcusmccurdy}
\href{http://github.com/volker48}{github.com/volker48}
\\
240 Montrose ST\sbull
Philadelphia, PA 19147\sbull

\spacedhrule{0.9em}{-0.4em}  % a horizontal line with some vertical spacing before and after

%Not currently using this section
%\roottitle{Summary}  % a root section title
%
%\vspace{-1.3em}  % some vertical spacing
%\begin{multicols}{2}  % open a multicolumn environment
%\noindent \emph{}
%\\
%\\
%%\end{multicols}
%
%
%\spacedhrule{0em}{-0.4em}
%
\roottitle{Experience}

\headedsection  % sets the header for the section and includes any subsections
  {\href{http://www.drexel.edu/}{Drexel University}}
  {\textsc{Philadelphia, Pennsylvania}} {%

  \headedsubsection  % sets the header for a subsection and contains usually body text
    {Research Engineer}
    {Jun\apo09 -- Present}
    {\bodytext{
    \begin{itemize}
        \item Performed background research and drafted a proposal for an 
            internal R\&D online learning project focusing on integrating 
            artificial intelligence into Drexel's online courses. Proposal after
            research was approximately \$700,000 under initial estimates and 
            was presented to the president of the university.
        \item Technical lead on a US Army funded Android development project. 
            Assessing the feasibility of and implementing agent based networking
            applications. Also created scenarios for testing and evaluation of
            networking protocols and performed the evaluations.
        \item Team lead in the design and implementation of a 
    distributed collaborative battle command planning system for the US army. 
    Oversaw and mentored junior developers during the project while working
    closely with the lead system integrator to deliver a premium product. 
    Integrated the final product with systems from French and German allies. 
    Extensive use of Swing and {NASA}'s {World Wind}.
\item Designed and implemented an {HTTP} proxy in {C\#} for a biometrics 
    fingerprint scanner and webserver. The proxy was for both the client
    application of the fingerprint scanner and the webserver for matching and
    enrolling biometrics data. The proxy enables the client and server to
    use an in house pure C content based networking middleware.
    \end{itemize}
}}
 }

\headedsection
  {\href{http://www.navsea.navy.mil/nswc/porthueneme/}{NSWC PHD}}
  {\textsc{Port Hueneme, California}} {%
  \headedsubsection
    {Combat System Engineer}
    {Jul\apo07 -- Jun\apo09}
    {\bodytext{
        \begin{itemize}
            \item Responsible for installation and configuration of remote 
    assistance systems on US Navy cruisers and destroyers. Remote system saved
    time and money by allowing shore based experts to provide remote support
to sailors both at sea and when docked. Resulted in a dramatic cost savings and
decreased the turn around time for repairs.
            \item Communicated directly with System Test Officer ({STO}) to
                ensure minimal or no down time while installing routers and 
                software for remote support.
        \end{itemize}
}}
}

\headedsection
  {\href{http://www.geeksquad.com/}{Geeksquad}}
  {\textsc{Westminster, California}} {%
  \headedsubsection
  {\acr{IT} Consultant}
    {Nov\apo09 -- Dec\apo09}
    {\bodytext{
        \begin{itemize}
    \item One of the first agents when Best Buy acquired Geeksquad. 
    \item Pioneered and assisted in the roll-out of processes and procedures 
        to the entire company.
    \end{itemize}

}}
}

\begin{center}
  \emph{Please refer to \href{http://www.linkedin.com/in/marcusmccurdy}{my Linkedin profile} for the complete list of work experiences.}
\end{center}


\spacedhrule{-0.2em}{-0.4em}

\roottitle{Education}

\headedsection
  {Drexel University}
  {\textsc{Philadelphia, Pennsylvania}} {%
  \headedsubsection
    {Masters degree in Computer Science}
    {2009 -- 2012}
    {\bodytext{Focused on Artificial Intelligence.}}
}

\headedsection
  {California State University at Long Beach}
  {\textsc{Long Beach, California}} {%
  \headedsubsection
    {Bachelors degree in Computer Engineering}
    {2001 -- 2007} 
    {\bodytext{Minor in Computer Science.}}
}

\spacedhrule{0.5em}{-0.4em}

\roottitle{Skills}

\inlineheadsection  % special section that has an inline header with a 'hanging' paragraph
  {Technical specialties:}
  {I have a passion for programming and often program just for fun. I've written the
  most code in Java followed by Python. While Python is currently my favorite
  I try to remain language agnostic and thoroughly enjoy learning new
  languages. I've also worked with both C and C++ including embedded programming
  in C on the 8051 microcontroller. Experience with web 
  technologies {HTML}, {CSS}, and {Javascript}. Extensive experience with
  {XML} and {JAXB}. Written the most {SQL} {PostgreSQL} and setup several 
  {PostgreSQL} installs. Linux system administration experience on my 
  own personal machine and servers for side projects. I use {Git} and 
  Mercurial daily and {SVN} weekly. I currently
  use IntelliJ as my IDE, but have used Visual Studio for {C\#} and {C++}
  development. I have also used {Netbeans} and {Eclipse} for Java development.
  My editor of choice is Vim despite attempts to coerce me into using Emacs.}

\inlineheadsection
  {Natural languages:}
  {English \emph{(native), German \emph{(limited)}.}}


\spacedhrule{1.6em}{-0.4em}

\roottitle{Interests}

\inlineheadsection
  {Non-exhaustive and in alphabetical order:}
  {Biking, cooking, Crossfit, gaming, hiking, Olympic lifting, programming, reading, running}


\end{document}
