% resume.sty

% Copyright (c) 2012 Cies Breijs
%
% The MIT License
%
% Permission is hereby granted, free of charge, to any person obtaining a copy
% of this software and associated documentation files (the "Software"), to deal
% in the Software without restriction, including without limitation the rights
% to use, copy, modify, merge, publish, distribute, sublicense, and/or sell
% copies of the Software, and to permit persons to whom the Software is
% furnished to do so, subject to the following conditions:
%
% The above copyright notice and this permission notice shall be included in
% all copies or substantial portions of the Software.
%
% THE SOFTWARE IS PROVIDED "AS IS", WITHOUT WARRANTY OF ANY KIND, EXPRESS OR
% IMPLIED, INCLUDING BUT NOT LIMITED TO THE WARRANTIES OF MERCHANTABILITY,
% FITNESS FOR A PARTICULAR PURPOSE AND NONINFRINGEMENT. IN NO EVENT SHALL THE
% AUTHORS OR COPYRIGHT HOLDERS BE LIABLE FOR ANY CLAIM, DAMAGES OR OTHER
% LIABILITY, WHETHER IN AN ACTION OF CONTRACT, TORT OR OTHERWISE, ARISING FROM,
% OUT OF OR IN CONNECTION WITH THE SOFTWARE OR THE USE OR OTHER DEALINGS IN THE
% SOFTWARE.

%
% This style contains some commands for making a LaTeX resume
%
% Please refer to the README.md file for more info.
%
% This project is currently hosted at: https://github.com/cies/resume
%




%%% LOAD AND SETUP PACKAGES



% To finetune lists with a inline heading and indented content.
% See the Experiences section in the example.
\usepackage{mdwlist}

% For multiple column text.
\usepackage{multicol}

% For \textscale, which I prefer over \sc (small caps).
% See the \acr command definition below.
\usepackage{relsize}

% Setup the hyperrefs witht he right color.
\usepackage{hyperref}  
\usepackage{xcolor}
\definecolor{dark-blue}{rgb}{0.15,0.15,0.4}
\hypersetup{colorlinks,linkcolor={dark-blue},citecolor={dark-blue},urlcolor={dark-blue}}

% XeTeX specific stuff with fall-back.
\usepackage{ifxetex}
\ifxetex

  \usepackage{fontspec}
  
  % the main font, with all features on
  \setmainfont
    [ ExternalLocation ,
      Mapping          = tex-text ,
      Numbers          = OldStyle ,
      Ligatures        = {Common,Contextual} ,
      BoldFont         = texgyrepagella-bold.otf ,
      ItalicFont       = texgyrepagella-italic.otf ,
      BoldItalicFont   = texgyrepagella-bolditalic.otf ]
    {texgyrepagella-regular.otf}
    
  % same like the main font, but without old-style nums
  \newfontfamily\newnums
    [ ExternalLocation ,
      Mapping          = tex-text ,
      Ligatures        = {Common,Contextual} ,
      BoldFont         = texgyrepagella-bold.otf ,
      ItalicFont       = texgyrepagella-italic.otf ,
      BoldItalicFont   = texgyrepagella-bolditalic.otf ]
    {texgyrepagella-regular.otf}
  % Comment out the previous statement and uncomment the following line to use the
  % Linux Libertine font (it has nice lignatures).
  % Make sure to have the `ttf-linux-libertine` package installed on Ubuntu.
%  \setmainfont[Mapping=tex-text, Numbers=OldStyle, Ligatures={Common,Contextual}]{Linux Libertine O}

  % needs an experimental and impossible to find package for xetex
  \usepackage[protrusion]{microtype}

\else

  % this case we likely use the `pdftex` back-end
  % therefor we lack:
  %  * lower case numbers,
  %  * ligatures and
  %  * some typographic niceties
  % We do make use of the possibility to use `microtype`
  \usepackage{tgpagella}
  \usepackage[expansion,protrusion]{microtype}

\fi



%%% DOCUMENT WIDE STYLING

\pagestyle{empty}
\setlength{\tabcolsep}{0em}
\xspaceskip7pt  % some more spacing between sentences (use "i.e.\ " or "with SQL\@. " in case of errors)



%%% CUSTOM COMMANDS

% main title (name) with subtitle (date)
\newcommand\maintitle[3]{\vbox to 0pt{\hfill\scriptsize\color{gray} #3}\vspace{-0.4em}\noindent{\LARGE \textbf{#1}}\ \ \ \emph{#2}}

% title for the root sections (experience, education, etc) of the resume
\newcommand*\roottitle[1]{\subsection*{#1}\vspace{-0.3em}\nopagebreak[4]}

% acr command, to quickly mark acronyms for special formatting
\newcommand*\acr[1]{\textscale{.85}{#1}}

% pretty bullet (created from a much smaller centerdot), \sbull contains its spacing
\newcommand*\bull{\raisebox{-0.365em}[-1em][-1em]{\textscale{4}{$\cdot$}}}
\newcommand*\sbull{\ \ \bull \ \ }

% it seems not to work when simply using \parindent...
\newlength{\newparindent}
\addtolength{\newparindent}{\parindent}

% a double \parindent...
\newlength{\doubleparindent}
\addtolength{\doubleparindent}{\parindent}
\addtolength{\doubleparindent}{\parindent}

% indentsection style, used for sections that aren't already in lists
% that need indentation to the level of all text in the document
\newenvironment{indentsection}%
{\begin{list}{}%
  {\setlength{\leftmargin}{\newparindent}\setlength{\parsep}{0pt}\setlength{\parskip}{0pt}\setlength{\itemsep}{0pt}\setlength{\topsep}{0pt}}%
}
{\end{list}}

% headerrow command, used for a new employer
\newcommand{\headedsection}[3]{\nopagebreak[4]\begin{indentsection}\item[]\textscale{1.1}{#1}\hfill#2#3\end{indentsection}\nopagebreak[4]}

% subheaderrow command, used for a new position
\newcommand{\headedsubsection}[3]{\nopagebreak[4]\begin{indentsection}\item[]\textbf{#1}\hfill\emph{#2}#3\end{indentsection}\nopagebreak[4]}

% body text (indented)
\newcommand{\bodytext}[1]{\nopagebreak[4]\begin{indentsection}\item[]#1\end{indentsection}\pagebreak[2]}

% \vspace variaties
\newcommand{\breakvspace}[1]{\pagebreak[2]\vspace{#1}\pagebreak[2]}
\newcommand{\nobreakvspace}[1]{\nopagebreak[4]\vspace{#1}\nopagebreak[4]}

% \spacedhrule a horizontal line with some vertical space before and after it
\newcommand{\spacedhrule}[2]{\breakvspace{#1}\hrule\nobreakvspace{#2}}

% \inlineheadsection command, used for a new employer
\newcommand{\inlineheadsection}[2]{\begin{basedescript}{\setlength{\leftmargin}{\doubleparindent}}\item[\hspace{\newparindent}\textbf{#1}]#2\end{basedescript}\vspace{-1.7em}}

% apo command, for an apostrophe that looks good on old style nums
\newcommand{\apo}{\raisebox{-.18ex}{'}{\hspace{-.1em}}}

% non space that allows line breaks
\newcommand*{\nsp}{\hskip0pt}



%%% MORE SPECIFIC COMMANDS

% CPP command (found it in some corner of the internet and decided to use it)
\newcommand{\CPP}{C\nolinebreak[4]\hspace{-.04em}\raisebox{.20ex}{\footnotesize\bf++}}

% LaTeX source of my resume
% =========================

% Commented for easy reuse... ;)

% See the `README.md` file for more info.

% This file is licensed under the CC-NC-ND Creative Commons license.


% start a document with the here given default font size and paper size
\documentclass[10pt,a4paper]{article}

\usepackage[a4paper,margin=0.75in]{geometry}

\usepackage[english]{babel}
\hyphenation{Some-long-word}

\usepackage{resume}


\begin{document}  % begin the content of the document
\sloppy  % this to relax whitespacing in favour of straight margins

\maintitle{Marcus McCurdy}{}% title on top of the document

\nobreakvspace{0.3em}  % add some page break averse vertical spacing

% \noindent prevents paragraph's first lines from indenting
% \mbox is used to obfuscate the email address
% \sbull is a spaced bullet
% \href well..
% \\ breaks the line into a new paragraph
\noindent\href{mailto:marcus.mccurdy@gmail.com}
{marcus.mccurdy\mbox{}@\mbox{}gmail.com}\sbull
\href{http://github.com/volker48}{github.com/volker48} \sbull
\href{http://www.linkedin.com/in/marcusmccurdy}
    {www.linkedin.com/in/marcusmccurdy}
\\
224 Virginia Ave\sbull
Haddon Township, NJ 08108\sbull \textsmaller{+1}714-791-8345

\spacedhrule{0.9em}{-0.4em}  % a horizontal line with some vertical spacing before and after

\roottitle{Summary}  % a root section title

\vspace{-1.3em}  % some vertical spacing
\begin{multicols}{2}  % open a multicolumn environment
\noindent \emph{Engineer at heart who has always loved 
seeing the joy building something useful can bring to
others.}
\\

I'm a passionate and pragmatic programmer who believes done is better than 
perfect while at the same time believing that software shouldn't be rushed. 
I enjoy writing correct code while maintaining a sense of urgency and momentum. 
I use testing to provide myself with a tight feedback loop and the freedom to 
refactor without fear. I thrive working in an environment where I can  
collaborate with other developers on different aspects of our project resulting
in a cohesive business feature that brings value to the company.

\end{multicols}


\spacedhrule{0em}{-0.4em}

\roottitle{Experience}

\headedsection
  {\href{http://www.50onred.com/}{50onRed}}
  {\textsc{Philadelphia, Pennsylvania} } 
  {%
 
  \headedsubsection
  {Lead Software Engineer}
  {Sept\apo12 -- Present}
  {\bodytext{
      \begin{itemize}
          \item Used PySpark to aggregate an analyze click and impression logs
                stored on Amazon S3.
          \item Provisioned and deployed an Apache Spark iPython notebook on 
                an Amazon EMR cluster.
          \item Lead developer on 50onRed's cross-browser JavaScript extension
              platform. A Python Flask web.
              app backed by a {MySQL} database and {SQLAlchemy} {ORM}. The
              Flask app is served  by Gunicorn and proxied by NGINX.
          \item Worked closely with operational analysts to provide custom
              reporting on 50's internal {PHP} dashboard and {MySQL} database.
          \item Updated a manual process of editing JavaScript in text fields
              and transformed it into a single page {AngularJS} application 
              providing a simple interface for creating ad units that are 
              compiled into fully functioning JavaScript ad units.
          \item Revolutionised 50's Python web scraping framework by rewriting
              a synchronous process into an asynchronous job queue with 
              \href{http://python-rq.org/}{RQ}.
          \item Major contributor to 50's Java and Jetty based popup serving
              daemon processing billions of requests daily.
          \item Implemented a concurrent Java application to process and 
              aggregate terabytes of {Akamai} log files daily and store them
              in a {PostgreSQL} database.
          \item Extensive {Amazon} {AWS} deployments and provisioning with
              {Ansible}.
          \item Installed and maintained a Sentry event logging 
              server to track errors across Java, Python, PHP, and 
              JavaScript code bases.
      \end{itemize}

    }
  }
  }
\headedsection
  {\href{https://github.com/volker48/}{Open Source Projects}}
  {\textsc{Philadelphia, Pennsylvania} } 
  {%
 
  \headedsubsection
  {Developer}
  {Sept\apo12 -- Present}
  {\bodytext{
      \begin{itemize}
          \item
              \href{https://chrome.google.com/webstore/detail/doge-pricer/oeaopbebinckgcdbihhdamchalemolpm}{DogePricer}
              a Chrome extension leveraging REST APIs to provide up to date
              prices of a cryptographic currency. It also provides setting
              custom price alerts and API polling frequency.
          \item
              \href{https://pypi.python.org/pypi/Flask-Mandrill/}{Flask-Mandrill}
              is a Flask extension that reduces the boilerplate needed to send
              email using Mandrill.
          \item \href{http://hashcolor.com/}{HashColor} a simple Python Flask
              web app that generates a unique hex color code for a given string. 
              This unique color can be used to provide a background color based
              on an application user's username or email address.
          \item \href{http://markovnamer.herokuapp.com/}{Namgen} a Python Flask
              app that utilizes NumPy and the NLTK names corpus to
              probabilistically create realistic names.


      \end{itemize}
  }
  }
  }

\headedsection  % sets the header for the section and includes any subsections
  {\href{http://www.drexel.edu/}{Drexel University}}
  {\textsc{Philadelphia, Pennsylvania}} {%

  \headedsubsection  % sets the header for a subsection and contains usually body text
    {Research Engineer}
    {Apr\apo09 -- Sept\apo12}
    {\bodytext{
    \begin{itemize}
        \item Performed background research and drafted a proposal for an 
            internal R\&D online learning project focusing on integrating 
            artificial intelligence into Drexel's online courses. Proposal after
            research was approximately \$700,000 under initial estimates and 
            was presented to the president of the university.
        \item Technical lead on a US Army funded Android development project. 
            Assessing the feasibility of and implementing agent based networking
            applications. Also created scenarios for testing and evaluation of
            networking protocols and performed the evaluations.
        \item Team lead in the design and implementation of a 
    distributed collaborative battle command planning system for the US army. 
    Oversaw and mentored junior developers during the project while working
    closely with the lead system integrator to deliver a premium product. 
    Integrated the final product with systems from French and German allies. 
    Extensive use of Swing and {NASA}'s {World Wind}.
\item Designed and implemented an {HTTP} proxy in {C\#} for a biometrics 
    fingerprint scanner and webserver. The proxy was for both the client
    application of the fingerprint scanner and the webserver for matching and
    enrolling biometrics data. The proxy enables the client and server to
    use an in house pure C content based networking middleware.
    \end{itemize}
}}
 }

\headedsection
  {\href{http://www.navsea.navy.mil/nswc/porthueneme/}{NSWC PHD}}
  {\textsc{Port Hueneme, California}} {%
  \headedsubsection
    {Combat System Engineer}
    {Jul\apo07 -- Apr\apo09}
    {\bodytext{
        \begin{itemize}
            \item Responsible for installation and configuration of remote 
    assistance systems on US Navy cruisers and destroyers. Remote system saved
    time and money by allowing shore based experts to provide remote support
to sailors both at sea and when docked. Resulted in a dramatic cost savings and
decreased the turn around time for repairs.
            \item Communicated directly with System Test Officer ({STO}) to
                ensure minimal or no down time while installing routers and 
                software for remote support.
        \end{itemize}
}}
}

\headedsection
  {\href{http://www.geeksquad.com/}{Geeksquad}}
  {\textsc{Westminster, California}} {%
  \headedsubsection
  {\acr{IT} Consultant}
    {Aug\apo02 -- Jul\apo07}
    {\bodytext{
        \begin{itemize}
    \item One of the first agents when Best Buy acquired Geeksquad. 
    \item Pioneered and assisted in the roll-out of processes and procedures 
        to the entire company.
    \end{itemize}

}}
}

%\begin{center}
%  \emph{Please refer to \href{http://www.linkedin.com/in/marcusmccurdy}{my Linkedin profile} for the complete list of work experiences.}
%\end{center}


\spacedhrule{-0.2em}{-0.4em}

\roottitle{Education}

\headedsection
  {Drexel University}
  {\textsc{Philadelphia, Pennsylvania}} {%
  \headedsubsection
    {Masters degree in Computer Science}
    {2009 -- 2012}
    {\bodytext{Focused on Artificial Intelligence.}}
}

\headedsection
  {California State University at Long Beach}
  {\textsc{Long Beach, California}} {%
  \headedsubsection
    {Bachelors degree in Computer Engineering}
    {2001 -- 2007} 
    {\bodytext{Minor in Computer Science.}}
}

\spacedhrule{0.5em}{-0.4em}

\roottitle{Skills}

\inlineheadsection  % special section that has an inline header with a 'hanging' paragraph
  {Technical specialties:}
  {I have a passion for programming and often program just for fun. I've written the
  most code in Python, Java, PHP, and JavaScript. While Python is currently my 
  favorite I try to remain language agnostic and thoroughly enjoy learning new
  languages and I've dabbled in {Go} and {Ruby}. I've also worked with both C and C++ including embedded programming
  in C on the 8051 microcontroller. Experience with web 
  technologies {HTML}, {CSS}, and {LESS}. Extensive experience with
  {XML} and {JAXB}. Written the most {SQL} for both {MySQL} and {PostgreSQL} and
  deployed both databases to {AWS} and {DigitalOcean}. Linux system
  administration (mostly Ubuntu) on {AWS} and my own personal machine. 
  I use {Git} and Mercurial daily. I currently use IntelliJ as my IDE, but have used Visual 
  Studio for {C\#} and {C++} development. I have also used {Netbeans} and 
  {Eclipse} for Java development. My editor of choice is Vim despite 
  attempts to coerce me into using Emacs.}

\inlineheadsection
  {Natural languages:}
  {English \emph{(native), German \emph{(limited)}.}}


\spacedhrule{1.6em}{-0.4em}

\roottitle{Interests}

\inlineheadsection
  {Non-exhaustive and in alphabetical order:}
  {Biking, cooking, cryptographic currencies, CrossFit, gaming, hiking, 
  Olympic lifting, programming, reading, and running.}


\end{document}
